\usepackage[spanish,es-tabla]{babel} % idioma: Español, no coloque nombre tablas como cuadro
\usepackage[T1]{fontenc} \usepackage[utf8]{inputenc} % símbolos especiales del idioma
\usepackage{times}
\usepackage[calc,showdow,spanish]{datetime2}
%\parindent = 0cm % configuración de sangría
\usepackage[backend=biber,style=ieee]{biblatex}
\usepackage{tabularx} % extra features for tabular environment
%\usepackage{amsmath} \usepackage{amssymb,amsfonts,latexsym,cancel}  % símbolos matemáticos 
%\usepackage{array}
%\usepackage{bm}
%\usepackage{epstopdf} % figuras en formato eps a pdf
\usepackage{hyperref} % agregar hiper enlaces dentro del archivo PDF generado
%\usepackage{longtable} % habilitar tablas largas
%\setcounter{MaxMatrixCols}{40} % configuración de limite columnas: 40
%\usepackage{multicol} % varias columnas al documento
%\usepackage{subfigure} % varias figuras
\usepackage[small,compact]{titlesec} \usepackage{titling} %cambiar el formato del titulo
%\newcolumntype{E}{>{$}c<{$}} % información de tablas en formato matemático
\usepackage{graphicx} % takes care of graphic including machinery
\usepackage{geometry} % configuración del dimensiones de la margen del documento 
\usepackage{filecontents}
\usepackage{pgfplots} % graficas de funciones matematicas
\pgfplotsset{compat=newest}
\usepgfplotslibrary{fillbetween}
\usepackage{siunitx}
\usepackage{pgfplotstable}
\usepackage{booktabs}
\usepackage{xcolor}
% \usepackage{LobsterTwo}
\usepackage{subcaption}
\usepackage{svg} % insertar imagenes .svg
\usepackage{tcolorbox}
\usepackage{fancyhdr} % configuración del formato del documento
\usepackage{authblk}
\usepackage[font=footnotesize]{caption}
\usepackage[toc,page]{appendix}
\usepackage{parskip}
\usepackage{amssymb, amsmath} % Paquetes matemáticos de la American Mathematical Society
\usepackage{float}
\usepackage{setspace}
\usepackage{parskip}
\usepackage{multirow}
\usepackage[all]{xy}
\usepackage{pstricks-add}
\usepackage{tikz}
\usepackage{circuitikz}
\usetikzlibrary{positioning,circuits.ee.IEC}
\usetikzlibrary{matrix}
\usetikzlibrary{calc}
\usetikzlibrary{fit}
\usetikzlibrary{shapes.geometric}
%\usepackage{showframe}


% *********************************************************
% *********************************************************
% ************* CONFIGURACIÓN DE PAQUETS ****************** 
% *********************************************************
% *********************************************************
%---------------------------------------------
% configuración formato de fecha
\DTMnewdatestyle{mydateformat}{%
	\renewcommand{\DTMdisplaydate}[4]{%
		%\DTMshortweekdayname{##4},\space% short weekday,
		%\DTMmonthname{##2}\nobreakspace%  (full) Month
		%\number##3,\space%                day,
		%\number##1%                       year
	}%
	\renewcommand{\DTMDisplaydate}{\DTMdisplaydate}%
}
%---------------------------------------------
% configuración de margen
\geometry{
	papersize = {216mm, 279.4mm},
	width = 20cm,
	height = 25cm,
	headsep = 5mm,
	head = 2cm,
	marginpar = 2mm,
	includeall,
}
%---------------------------------------------
% configuración de hiper vínculos
\hypersetup{
	colorlinks=true,       % false: boxed links; true: colored links
	linkcolor=black,        % color of internal links
	citecolor=black,        % color of links to bibliography
	filecolor=magenta,     % color of file links
	urlcolor=black
}
%---------------------------------------------
% configuración de estilo de la margen
\fancyhf{}
\renewcommand{\headrulewidth}{0pt}
\fancyhead[LO,L]{
	\includegraphics[scale=0.119]{img/escudo.jpg}
}
\fancyhead[RO,R]{
	\fontsize{9}{9}
	\textsf{
		Modulación y demulación a una señal\\
		Teoría de telecomunicaciones I, Grupo A12\\
		\vspace{-1.3mm}
		\today
		\DTMsetdatestyle{mydateformat}
		\today
	}
}
\fancyfoot[C]{\thepage}
\pagestyle{fancy}
%---------------------------------------------
\spanishdecimal{.}
\tcbuselibrary{theorems}
%---------------------------------------------
% configuración para sección de archivos e información adicional al documento
\addto\captionsspanish{
	\renewcommand\appendixname{Anexo}
	\renewcommand\appendixpagename{Anexos}
}
%---------------------------------------------
% algunas configuraciones del cuerpo del documento
\renewcommand{\tablename}{Tabla}
\renewcommand{\baselinestretch}{0.8} % Para indicar el tamaño del entrelineado
\titleformat{\subsection}[wrap]
{\large\normalfont\fontseries{b}\selectfont\filright}
{\thesubsection.}{.5em}{}
\titlespacing{\subsection}
{12pc}{1.5ex plus .1ex minus .2ex}{1pc}

\titleformat{\section}[wrap]
{\large\normalfont\fontseries{b}\selectfont\filright}
{\thesection.}{.5em}{}
\titlespacing{\section}
{12pc}{1.5ex plus .1ex minus .2ex}{1pc}
\setlength{\parskip}{1.5mm} % Modificar espacio entre párrafos
\renewcommand*{\bibfont}{\footnotesize} % Cambiar tamaño bibliografía
%---------------------------------------------
% configuración de estilo para autores y afiliaciones
\renewcommand*{\Authsep}{ y }
\renewcommand*{\Authand}{ y }
\renewcommand*{\Authands}{, }
\renewcommand*{\Affilfont}{\normalsize}
%\renewcommand*{\Authfont}{\bfseries}    % make author names boldface    
\setlength{\affilsep}{-2mm}   % set the space between author and affiliation
\renewcommand\Authfont{\fontsize{12}{12}\selectfont} % Cambiar tamaño de letra autores
\renewcommand\Affilfont{\fontsize{9}{9}\itshape} % Cambiar tamaño de letra afiliaciones de autores
%---------------------------------------------
% Simbolo fuente AC
\tikzset{circuit declare symbol = ac source}
\tikzset{set ac source graphic = ac source IEC graphic}
\tikzset{
	ac source IEC graphic/.style={
			transform shape,
			circuit symbol lines,
			circuit symbol size = width 3 height 3,
			shape=generic circle IEC,
			/pgf/generic circle IEC/before background={
					\pgfpathmoveto{\pgfpoint{-0.8pt}{0pt}}
					\pgfpathsine{\pgfpoint{0.4pt}{0.4pt}}
					\pgfpathcosine{\pgfpoint{0.4pt}{-0.4pt}}
					\pgfpathsine{\pgfpoint{0.4pt}{-0.4pt}}
					\pgfpathcosine{\pgfpoint{0.4pt}{0.4pt}}
					\pgfusepathqstroke
				}
		}
}
%---------------------------------------------
