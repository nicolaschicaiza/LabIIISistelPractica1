\section{Introducción}%
\label{sec:introduccion}

El informe del Trabajo 5 de la asignatura Teoría de las Telecomunicaciones 1 se desarrolla en este documento. El trabajo consiste en considerar una señal planteada de tipo diente de sierra no periódica para aplicar los teoremas de modulaciones lineales.

Iniciar con el desarrollo analítico es necesario debido a que para alcanzar los resultados esperados en la simulación, se requiere conocer de antemano los efectos que representa al aplicar los conceptos de Modulaciones Lineales a la señal, esto permitirá comprobar los resultados obtenidos por medio de un algoritmo realizado en MATLAB.

El objetivo de dicha simulación al aplicar los conceptos del tema es realizar una modulación y demodulación de la señal a partir de la Modulación de Doble Banda Lateral con Portadora Suprimida(DSBSC), el cual es un esquema de modulación lineal que consiste en trasladar directamente el espectro de la señal mensaje hasta una frecuencia portadora.

Además, para el proceso de recuperación de la señal se debe introducir en simulación errores de sincronismo que se pueden obtener al realizar cambios en los valores de frecuencia y fase de la señal portadora en el receptor. Para cada posible escenario de estos efectos se analiza lo que le ocurre a la señal recuperada. Es posible que al introducir estos efectos de errores en la señal recuperada la señal mensaje tendrá una  perdida de la información transmitida.
